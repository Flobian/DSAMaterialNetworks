
 \exercise[%
 topic = Numerisches L\"osen von Differentialgleichungen
  ]


Oft ist es nicht m\"oglich Differentialgleichungen (DGL) analytisch zu l\"osen. Dann m\"ussen wir auf numerische L\"osungsverfahren zur\"uckgreifen. Eines der einfachsten Verfahren ist das sogenannte \emph{Euler-Verfahren}, auch intuitiv \emph{Methode der kleinen Schritte} genannt.

Sei folgende DGL gegeben 
\begin{align}
\dot y = f(t,y)\,.
\end{align}
mit dem Anfangswert $y(t_0)=y_0$. Dann berechnen wir iterativ aus dem derzeitigen Zustand $(t_k,y_k)$ den jeweils folgenden Wert $y_{k+1}$ als
\begin{align}
\dot y_{k+1} =y_k + h\cdot f(t_k,y_k)\,.
\end{align}
Dabei handelt es sich um eine \emph{lineare} Approximation und $h$ gibt die Schrittweite an. In der Regel resultieren kleinere Schrittweiten $h$ in besseren numerischen L\"osungen.

 \subexercise[%
  topic={Exponentielles Wachstum},
    ]
		
Zun\"achst wollen wir uns mit einem eindimensionalen Problem besch\"aftigen und zwar mit dem exponentiellen Wachstum.

Wie lautet die analytische L\"osung der folgenden DGL?
\begin{align}
\dot x(t) = \lambda x(t)\\
x(t_0)=x_0
\label{eqn:exponentielles}
\end{align}

\subexercise[%
  topic={Eindimensionales Euler-Verfahren},
    ]
L\"ose die obige DGL numerisch mit dem Euler-Verfahren. Setze zun\"achst $h=1$, $x_0=3$ und $\lambda=1.2$.

Stelle nun den zeitlichen Verlauf grafisch dar und vergleiche mit der analytischen L\"osung. Wie verh\"alt sich das System wenn die Schrittweite stark erh\"oht wird?\\
Zusatz: Variiere $\lambda \in \{-2,-0.1,0.1,2\}$. Stelle die verschiedenen Kurven $x(t)$ dar, wie verhalten sie sich? 

\subexercise[%
  topic={Numerisches \"uberpr\"ufen der Stabilit\"at von Fixpunkten},
    ]
Bestimme f\"ur die obige DGL alle Fixpunkte $x^{*}$ analytisch. Nun wollen wir die Stabilit\"at \"uberpr\"ufen. Dazu lenken wir die DGL ein wenig von dem Fixpunkt aus und \"uberpr\"ufen numerisch ob sich das System wieder zur\"uck zum Fixpunkt bewegt oder sich davon entfernt.
\"uberpr\"ufe dies f\"ur $\lambda = \{-1,0,1\}$.

\subexercise[%
  topic={Zusatz: Fehlerabsch\"atzung des Euler Verfahren},
    ]

Da wir die analytische L\"osung $x_{\mathrm{ana}}$ der obigen DGL kennen k\"onnen wir \"uberpr\"ufen wie sich die Abweichung (oder auch der Fehler) in Abh\"angigkeit von der Schrittweite $h$ verh\"alt.

Stelle hierf\"ur $\mathrm{Fehler}(h)=|x_{\mathrm{ana}}(t) - x_{\mathrm{numerisch}}(t,h)|$ dar. W\"ahle zum Beispiel $t=100$ und variiere die Schrittweite $h$. Wie sieht der Zusammenhang $\mathrm{Fehler}(h)$ aus?

 \subexercise[%
  topic={Numerische L\"osung der Lotka-Volterra Gleichungen},
    ]

Wir k\"onnen das Euler-Verfahren auch auf zweidimensionale DGLs anwenden. Hierzu w\"ahlen wir als Beispiel die \emph{Lotka-Volterra} Gleichungen die wir bereits im Unterricht analytisch besprochen haben.

Dieses R\"auber-Beute System wird wie folgt beschrieben:
\begin{align}
\dot x(t) = x(3-x-2y)\\
\dot y(t) = y(2-x-y)\,.
\end{align}
Das Euler-Verfahren funktioniert \"ahnlich wie im eindimensionalen Fall:
\begin{align}
x_{k+1} = x_{k} + \dot x(t)_{k} \\
y_{k+1} = y_{k} + \dot y(t)_{k}\,.
\end{align}
Nutze das Euler-Verfahren um das Lotka-Volterra System zu l\"osen. Nutze verschiedene Anfangsbedingungen und analysiere zu welchen der analytisch beobachteten Fixpunkten sich das System hin bewegt. Stelle hierf\"ur $x(t)$ und $y(t)$ dar. Welche der analytisch vorhergesagten Fixpunkte werden nicht beobachtet?

\subexercise[%
  topic={Zusatz: Oszillationen im Lotka-Volterra System},
    ]
Eine leicht ver\"anderte Variante der Lotka-Volterra Gleichungen lautet
\begin{align}
\dot x(t) = x(\alpha -\beta y)\\
\dot y(t) = y(\gamma -\delta y)\,.
\end{align}
Stelle verschiedene \emph{Trajektorien} $(x(t),y(t))$ grafisch dar. Varriiere daf\"ur die Parameter $\alpha$, $\beta$, $\gamma$ und $\delta$. Wie verh\"alt sich das System wenn du zum Beispiel $\alpha=2/3$, $\beta=4/3$, $\gamma=1$ und $\delta=1$ und $x_0=y_0=0.9$ verwendest?

\subexercise[%
  topic={Zusatz: \emph{Basins of Attraction} Lotka-Volterra System},
    ]
Recherchiere was die \emph{Basins of Attraction} sind und ermittle diese numerisch f\"ur eine bestimmte Parameter Wahl f\"ur die Lotka-Volterra Gleichungen.
