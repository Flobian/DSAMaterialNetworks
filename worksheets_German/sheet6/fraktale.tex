
\exercise[
    topic = Fraktale und Selbst\"ahnlichkeit
]

Saskia und Konrad haben beide die Koch-Kurve eingef\"uhrt, Konrad als 
Lindenmayer-System und Saskia als Fraktal. Wir wollen nun dieses Fraktal
f\"ur endliche 
Iterationsschritte $n$ zeichen. Es wird 
bestimmt durch das Lindenmayer-System
\begin{subequations}
\label{eq:Koch-L}
\begin{align}
    \mathrm{Variablen} &= \left\{F\right\} \\
    \mathrm{Konstanten} &= \left\{+,-\right\} \\
    \mathrm{Produktionsregeln} &= \left\{F\rightarrow F-F++F-F\right\} \\
    \mathrm{Axiom} &= F.
\end{align}
\end{subequations}
Hier bedeutet $F$, dass eine \emph{turtle} (bzw. ein Zeichenstift) eine
bestimmte Strecke vorw\"arts geht, $-$, dass sich die \emph{turtle} um
$60^\circ$ nach links dreht und $+$, dass sich die \emph{turtle} um
$60^\circ$ nach rechts dreht.

\subexercise[
    topic = Anwenden der Iteration in L-System
]
    \label{ex:L-Koch}
    Schreibe eine Funktion in Python, die f\"ur die Koch-Kurve die
    Produktionsregeln $n$ mal auf das Axiom anwendet und so ein Wort produziert, mit
    dem einer \textit{turtle} vermittelt werden kann, wie gezeichnet werden soll.

\subexercise[
    topic = Teenage Mutant L-System Turtle
    ]

    Das folgende Programm bietet einen Einstieg in das Zeichnen mit
    einer \textit{turtle}. \footnote{Eine ausf\"uhrliche Dokumentation
        der \textit{turtle}-Befehle findet ihr auf
        \url{https://docs.python.org/2/library/turtle.html}
    }
        Spiele mit den Befehlen im ersten Teil des Codes herum. Was
        macht die Funktion im letzten Teil des Codes?

    \lstinputlisting{./code/turtleexample.py}

\subexercise[
    topic = Koch-Kurve zeichnen
    ]

    Schreibe eine Funktion, die das Ergebnis der $n$-fachen Iteration
    des L-Systems (\ref{eq:Koch-L}) mithilfe der \emph{turtle} zeichnet.
    Beachte, dass der Befehl ausgel\"ost durch $F$ skaliert werden muss
    mit variierendem $n$ (d.h. dass der forward-Befehl mit $1/3^n$ der
    urspr\"unglichen L\"ange aufgerufen werden muss).

    Wie muss das Axiom ver\"andert werden, damit statt der Koch-Kurve
    die Koch'sche Schneeflocke gezeichnet wird?

\subexercise[
    topic = Allgemeine L-Systeme
]
\label{ex:L-allg}
Verallgemeinere deine Funktion aus Aufgabe \ref{ex:L-Koch}, sodass beliebige
Lindenmayer-Systeme produziert werden, die $n$-fach iteriert werden.

\exercise[
    topic = Fraktale erzeugen
]
Betrachte das L-System
\begin{subequations}
    \label{eq:Sierpinski-L}
\begin{align}
    \mathrm{Variablen} &= \left\{A,B\right\} \\
    \mathrm{Konstanten} &= \left\{+,-\right\} \\
    \mathrm{Produktionsregeln} &= \left\{A\rightarrow +B-A-B+,\
        B\rightarrow -A+B+A-\right\} \\
    \mathrm{Axiom} &= A.
\end{align}
\end{subequations}
Hier stehen sowohl $A$ als auch $B$ daf\"ur, dass die \emph{turtle}
vorw\"arts l\"auft. Bei $-$ soll sie sich um $60^\circ$ nach links
drehen, bei $+$ um $60^\circ$ nach rechts.

Welches Fraktal wird erzeugt?

\subexercise[  
topic = L-Systeme mit Speicher
]
Bei Systemen mit Speicher merkt sich die \emph{turtle} ihre aktuelle
Position im zweidimensionalen Raum, sobald sie in der Zeichenkette eine Klammer $[$ findet. Sie
folgt dann weiter den Befehlen in der Zeichenkette, bis sie auf das Ende
des momentanen Befehls trifft, markiert durch $]$. Dann springt sie auf
die Position im Raum zur\"uck, die sie bei $[$ hatte und folgt weiter
der Zeichenkette. Betrachte das L-System
\begin{subequations}
    \label{eq:Farn-L}
\begin{align}
    \mathrm{Variablen} &= \left\{F\right\} \\
    \mathrm{Konstanten} &= \left\{+,-,[,]\right\} \\
    \mathrm{Produktionsregeln} &= \left\{ F \rightarrow
        F[+F]F[-F]F\right\}\\
\mathrm{Axiom} &= F.
\end{align}
\end{subequations}
Der Winkel, den $+,-$ beschreiben, soll diesmal $180^\circ/7$ betragen.
Welchem Objekt \"ahnelt das Ergebnis?
