\documentclass[a4paper]{exercisesheet}

%\usepackage{mathpazo}
%\usepackage{mathptmx}
%\usepackage{newtxmath}
\usepackage[T1]{fontenc}
\usepackage[utf8]{luainputenc}
\usepackage[charter]{mathdesign}
\let\sfdefault=\rmdefault
\def\ttdefault{txtt}

\usepackage{graphicx}

\usepackage[dvipsnames]{xcolor}
%\colorlet{maincolor}{blue!50!black}
\colorlet{maincolor}{black}

\usepackage{listings}
\usepackage{amsmath}
%\lstset{
%  frame=lines,
%  backgroundcolor=\color{maincolor!15},
%  rulecolor=\color{maincolor},
%  language=Python
%  keywordstyle=\bfseries\color{maincolor},
%  numbers=left,
%  numberstyle=\scriptsize\color{maincolor!70},
%}

\linespread{1.04}

\usepackage[english]{babel}

\usepackage{blindtext}

\sheetconf{
    lecture   = {Networks and Complex Systems},
  lecturer  = {F.~Klimm~und~B.F.~Maier},
  semester  = {Sch\"ulerakademie 5.2 (Ro\ss leben 2016)},
  author    = {},
  % teacher,
  solutions=false,
}

\setsheetfont{lecture on titlepage}{\sffamily\Huge}
\setsheetfont{sheet title}{\sffamily\Large}
\setsheetfont{type on titlepage}{\sffamily\scriptsize\color{maincolor}}
\setsheetfont{sheet topic}{\sffamily\Huge\color{maincolor}}
\setsheetfont{sheet lecture}{\it\sffamily\Large\color{maincolor}}
\setsheetfont{exercise topic}{\sffamily\Large\color{maincolor}}
\setsheetfont{exercise label}{\sffamily\Large\color{maincolor}}
\setsheetfont{subexercise topic}{\sffamily\large\color{maincolor}}
\setsheetfont{subexercise label}{\sffamily\large\color{maincolor}}

\setsheettemplate{sheet title (student)}{Worksheet~\thesheet}
\setsheettemplate{exercise name}{Exercise}
\setsheettemplate{subexercise name}{Subexercise}

\lstset{%
  %linewidth=\textwidth,
  %linewidth=16cm,
  language=Python,                  % the language of the code
  basicstyle=\ttfamily\small,
  backgroundcolor=\color{maincolor!5},
  %basicstyle=\footnotesize,      % the size of the fonts that are used for the code
  numbers=left,                   % where to put the line-numbers
  stepnumber=1,                   % the step between two line-numbers. If it's 1, each line 
                                  % will be numbered
  numberstyle=\scriptsize\color{maincolor!70},
  numbersep=5pt,                  % how far the line-numbers are from the code
  frame=single,                   % adds a frame around the code
  rulecolor=\color{black},        % if not set, the frame-color may be changed on line-breaks
  tabsize=4,                      % sets default tabsize to 2 spaces
  captionpos=b,                   % sets the caption-position to bottom
  breaklines=true,                % sets automatic line breaking
  breakatwhitespace=false,        % sets if automatic breaks should only happen at whitespace
  %keywordstyle=\color{blue},      % keyword style
  %commentstyle=\color{dkgreen},   % comment style
  %stringstyle=\color{colorNavy},  % string literal style
  morekeywords={*,with, where, from, union, all, as},
  extendedchars=true,
  literate={ä}{{\"{a}}}1 {ö}{{\"o}}1 {ü}{{\"u}}1,
}


\usepackage{pstricks,pst-node,pst-tree}


\begin{document}

  \sheet[%
  number=5,
  topic={Dynamical Systems},
      %deadline=Deadline: \today,
    ]

\vspace{-1cm}
\noindent\rule{12cm}{0.4pt}

  \exercise[%
  topic = Population Dynamic for Discrete Time Steps
  ]


In the morning we discussed the population dynamic after Verhulst. This map is given by

  \begin{equation}
      x_{n+1} = \lambda x_n(1-x_n)\,.
  \end{equation}


  \begin{enumerate}
      \item If we assume $x\in [0,1]$, what does this indicate for the strength of the growth parameter $\lambda$?
      \item What are the fixed points $x^*$ of this dynamic?
      \item What is the stability of the smallest fixed point for $\lambda<1$ and $\lambda>1$? To analyse this, look at behaviors for values $x^*+\epsilon$ (with $\epsilon\ll1$)). The object $\epsilon$ is also called a \emph{perturbation}. We can investigate the stability of fixed points by analysing how the system behaves under small pertubations, for example  $\epsilon=0.01$.
  \end{enumerate}
    
  \exercise[%
  topic = Dynamic System in Continuous Time 
  ]


A class of dynamical systems in continuous time might be described as 
  \begin{equation}
      \dot x = f(x)\,,
  \end{equation}
 where the function $f:X\rightarrow X$ is a one-dimensional map. In such systems the fixed points $x^*$  are given by 
    \begin{equation}
      f(x^*) = \dot x|_{x=x^*} = 0\,.
  \end{equation}
  
  \subexercise

Analyse the dynamical system
  \begin{equation}
      \dot x = x^2 -1\,.
  \end{equation}
  
  \begin{enumerate}
      \item Find the fixed points $x^*$ and 
      \item estimate their stability with the vector field method.
  \end{enumerate}

  \subexercise
  
  Find fixed points $x^*$  and their stability for the following dynamical systems:
  \begin{enumerate}
      \item $\dot x = - x^3$
      \item $\dot x = x^3$
      \item $\dot x =x^2$
      \item $\dot x = x$
      \item $\dot x = 0$
      \item $\dot x = x-x^3$
  \end{enumerate}
  
  \subexercise[topic=Bifurcation]
  
  
  We investigate the dynamical system
  \begin{equation}
      \dot x = x^2 - a,
  \end{equation}
  with parameter $a\in I\!\!R$.
  Estimate the fixed points $x^*$ and their stability. How many fixed points exist for different values of $a$ and how is their stability affected?

  \subexercise[topic=Newton Friction]
  
  
  The friction experienced by falling objects in air can be described as
  $F(v) = \beta v^2$, where $\beta$ is the \emph{friction constant} and
  $v$ the velocity of the falling object of mass $m$. The gravitational
  acceleration of the earth is $g$. The force is directed in positive $z$ direction, thus upwards and against the falling direction. Accordingly, the temporal change of the velocity can be described by 
    \begin{equation}
      m\dot v = -mg+\beta v^2\,.
  \end{equation}
 
 Find the fixed point $v^*$ and its stability. How can you interpret this physically and why are the results intuitive? 
  
  
  A general form of the temporal dynamic of the velocity $v$ is 
  \begin{equation}
      m\dot v = -mg-\mathrm{sgn}(v)\beta v^2\,,
  \end{equation}
with the \emph{sign} function
  \begin{equation}
      \mathrm{sgn}(x)=\begin{cases}-1 & x<0\\
          0 & x=0\\
          +1 & x>0\end{cases}\,.
  \end{equation}
  This indicates that the friction is always opposed to the movement.
  Estimate the equilibrium  $v^*$ for $g=0$. Is this fixed point
  stable? In what physical situation is $g=0$ a good estimation and to what extent is the stability analysis reasonable?
  
 \exercise[topic=Linear Stability
 ]
  \subexercise
  
  
  Discuss the dynamical system
  \begin{equation}
      \dot x = \lambda x
  \end{equation}
  with $\lambda\in I\!\!R$. Which natural system can be well approximated by this equation?

Solve the differential equation with the method of separation of variables (look at the approach we used for the contagion dynamic). Does the system have a fixed point $x^*$? If so, for which values of   $\lambda$ does it exist?

  \subexercise
  
  We discuss an arbitrary differentiable function $f(x)$. Show that the function can be approximated linearly at any point $x_0$ by 
    \begin{equation}
      f(x) \approx f(x_0) + m(x-x_0)\,.
  \end{equation}
  Show furthermore, that the slope of the function is given by $m=f'(x_0)$, where $f'(x)$ is the first derivative of the function $f(x)$. 

  \subexercise
  
  Show, without graphical methods, that the dynamical system 
  
  \begin{equation}
      \dot x = x-x^3
  \end{equation}
is linearly stable at the points $x^*_1=-1$ and $x^*_3=1$.

  \exercise[topic=Eigenwerte und Eigenvektoren]
  \subexercise
We investigate the matrix
  \begin{equation}
      \hat A = \begin{pmatrix} 1 & 1 \\ 0 & 2\end{pmatrix}\,.
  \end{equation}
What are its eigenvalues and eigenvectors $\vec w_1$ and
  $\vec w_2$?
  
%  \subexercise
%  Seien die Koordinaten $(x,y)$ die Beschreibung des zweidimensionalen
%  Raumes, der durch die Vektoren 
%  \begin{equation}
%      \vec e_1 = \begin{pmatrix} 1\\0\end{pmatrix}\qquad
%      \mathrm{und}\qquad 
%      \vec e_2 = \begin{pmatrix} 0\\1\end{pmatrix}
%  \end{equation}
%  aufgespannt wird. Gegeben sei nun die Matrix 
%  \begin{equation}
%      \hat B = \begin{pmatrix} 1 & 2 \\ 2 & 1\end{pmatrix}
%  \end{equation}
%  aus der Vorlesung. Zeige, dass in den Koordinaten $(\tilde x,\tilde
%  y)$, die nat\"urlich f\"ur die Matrix $\hat B$ sind (d.h. in Form der
%  Eigenvektoren der Matrix ausgedr\"uckt werden k\"onnen), gilt
%  \begin{document}
%  \begin{pmatrix} \tilde x\\ \tilde y \end{pmatrix}
%  = \begin{pmatrix} 1\\  0 \end{pmatrix} + \begin{pmatrix}
%      \tilde x\\ \tilde y \end{pmatrix}
\end{document}
