\documentclass[a4paper]{exercisesheet}

%\usepackage{mathpazo}
%\usepackage{mathptmx}
%\usepackage{newtxmath}
\usepackage[T1]{fontenc}
\usepackage[utf8]{luainputenc}
\usepackage[charter]{mathdesign}
\let\sfdefault=\rmdefault
\def\ttdefault{txtt}

\usepackage{graphicx}

\usepackage[dvipsnames]{xcolor}
%\colorlet{maincolor}{blue!50!black}
\colorlet{maincolor}{black}

\usepackage{listings}
\usepackage{amsmath}
%\lstset{
%  frame=lines,
%  backgroundcolor=\color{maincolor!15},
%  rulecolor=\color{maincolor},
%  language=Python
%  keywordstyle=\bfseries\color{maincolor},
%  numbers=left,
%  numberstyle=\scriptsize\color{maincolor!70},
%}

\linespread{1.04}

\usepackage[english]{babel}

\usepackage{blindtext}

\sheetconf{
    lecture   = {Networks and Complex Systems},
  lecturer  = {F.~Klimm~und~B.F.~Maier},
  semester  = {Sch\"ulerakademie 5.2 (Ro\ss leben 2016)},
  author    = {},
  % teacher,
  solutions=false,
}

\setsheetfont{lecture on titlepage}{\sffamily\Huge}
\setsheetfont{sheet title}{\sffamily\Large}
\setsheetfont{type on titlepage}{\sffamily\scriptsize\color{maincolor}}
\setsheetfont{sheet topic}{\sffamily\Huge\color{maincolor}}
\setsheetfont{sheet lecture}{\it\sffamily\Large\color{maincolor}}
\setsheetfont{exercise topic}{\sffamily\Large\color{maincolor}}
\setsheetfont{exercise label}{\sffamily\Large\color{maincolor}}
\setsheetfont{subexercise topic}{\sffamily\large\color{maincolor}}
\setsheetfont{subexercise label}{\sffamily\large\color{maincolor}}

\setsheettemplate{sheet title (student)}{Worksheet~\thesheet}
\setsheettemplate{exercise name}{Exercise}
\setsheettemplate{subexercise name}{Subexercise}

\lstset{%
  %linewidth=\textwidth,
  %linewidth=16cm,
  language=Python,                  % the language of the code
  basicstyle=\ttfamily\small,
  backgroundcolor=\color{maincolor!5},
  %basicstyle=\footnotesize,      % the size of the fonts that are used for the code
  numbers=left,                   % where to put the line-numbers
  stepnumber=1,                   % the step between two line-numbers. If it's 1, each line 
                                  % will be numbered
  numberstyle=\scriptsize\color{maincolor!70},
  numbersep=5pt,                  % how far the line-numbers are from the code
  frame=single,                   % adds a frame around the code
  rulecolor=\color{black},        % if not set, the frame-color may be changed on line-breaks
  tabsize=4,                      % sets default tabsize to 2 spaces
  captionpos=b,                   % sets the caption-position to bottom
  breaklines=true,                % sets automatic line breaking
  breakatwhitespace=false,        % sets if automatic breaks should only happen at whitespace
  %keywordstyle=\color{blue},      % keyword style
  %commentstyle=\color{dkgreen},   % comment style
  %stringstyle=\color{colorNavy},  % string literal style
  morekeywords={*,with, where, from, union, all, as},
  extendedchars=true,
  literate={ä}{{\"{a}}}1 {ö}{{\"o}}1 {ü}{{\"u}}1,
}


\begin{document}

  \sheet[%
  number=1,
      topic={Introduction to Network Theory},
      %deadline=Deadline: \today,
    ]

\vspace{-1cm}
\noindent\rule{12cm}{0.4pt}

  \exercise[%
  topic = Create Graphs with Python
    ]

We want to use the coding language Python to create graphs and analyse them. The function {\tt pathgraph} in the file {\tt pathgraph.py} creates the adjacency matrix of the \emph{path graph}.

  
\lstinputlisting{./code/pathgraph.py}



 \subexercise[%
  topic=Understanding the Program and Adapting it,
    ]

Go line by line through the code shown above to understand what is happening. Particularly answer the following questions:

\begin{enumerate}
\item What are the input and output of the function {\tt pathgraph( n )}?
\item How does the adjacency matrix look like at each step of the
    FOR-Loop? Check your conjecture by inserting the command {\tt
        print(A)} after line $17$.
\item Why does the FOR-loop go from $0$ to $n-1$?
\item What happens in line $22$? What would happen without this command?
    Check your conjecture by commenting out this line.
\end{enumerate}

The benefit of functions is that we can reuse them mutliple times in a program. Extend the above program such that it prints not only the adjacency matrix of a path with ten but also fifteen nodes.

 \subexercise[%
  topic=Creating Graphs,
    ] \label{subex:creating}

Create functions that create other synthetical graphs consisting of $n$ nodes. You can use the function {\tt pathgraph( n )} as a starting point and adapt it.

\begin{enumerate}
\item null graph, also called empty graph
\item cycle graph
\item complete graph
\item complete bipartite graph with group sizes $m$ und $n$
\end{enumerate}


\exercise[%
  topic=Illustrate Graphs with Python
    ]


Usually, we represent graphs as adjacency matrices. This is useful for their further mathematical analysis. Sometimes is it appropriate to visualise them with the help of a computer. One tool for this is the library {\tt networkx}. Below is a small example of a program that creates a path graph and its visualisation.

\lstinputlisting{./code/draw_graph.py}

First, we have to load the necessary libraries with the {\sc import}
command. Second, we create the adjacency matrix of a path graph and
convert it to a `networkx' object. Finally, this can be used by the
fucntion {\tt draw()}  to create an illustration.

Now, adopt this program to illustrate the other networks from Subexercise~\ref{subex:creating} in size $n=8$.

  \exercise[%
  topic=Investigating Graphs with Python
    ]
Now we will use the functions from Subexercise~\ref{subex:creating} to analyse the networks by looking at some simple network properties.

 \subexercise[%
  topic=Counting Edges,
    ]

We want the investigate the relationship between the number $n$ of nodes in a network with the number $m$ of edges in the discussed synthetic networks. To achieve this go through the following steps:


\begin{enumerate}
\item Create a function that takes the adjacency matrix of a network as
    an input and returns the number of edges as output. (\emph{Hint: The numpy-function {\tt sum()} could be useful.})
\item Use a FOR-loop to create networks of varying size $n$ and count
    their number of edges. Save the results into a vector.
\item  Use {\tt matplotlib} to illustrate the relationship $m(n)$ for the different network models
\item Compare these numerical results with the analytical formulas we derived earlier during the lecture.
\end{enumerate}


 \subexercise[%
  topic=Comparing Synthetic with Empiric Networks,
    ]


So far we focussed on network models. Now we want to compare these with real (empirical) networks that were created from data.

We can use the function {\tt loadtxt} to load the adjacency matrix from a text file.

\begin{lstlisting}
H = np.loadtxt("hens.txt")
\end{lstlisting}

The network carries the information which of the $32$ chickens is fighting against which other ones.

Count the number of edges in this network, illustrate it, and compare it with the network models we introduced earlier. Which model describes the data best?


Repeat this for a second network {\tt florence.txt}. Is one of the
networks able to describe the data well? This network describes the
marriage relationships between the noble families in Florence in the
13th century. In the file {\tt florence\_families.txt}  you find the
names of the families (the line number corresponds with the node
number). Illustrate the network and colour-code each node with its
degree $k_i$ and a label giving its name. Which family has the most
connections?



\exercise[%
  topic=Optional Tasks
    ]

\subexercise[%
  topic=Multipartite Networks,
    ]

Similar to bipartite networks with two groups of nodes there exist
\emph{multipartite} networks, also called $p$-partite networks, which
consist of $p$ groups of nodes. Edges are only created between the
groups but never inside of them. If all possible edges exist we call
them \emph{complete multipartite graphs}. Create a function that creates
such $p$-partite networks with each of the $p$ groups having the same
size $g$.

Find analytic formulas for the number $n$ of nodes in the whole graph and the number $m$ of edges. Check your results numerically and illustrate the network.
		
		\subexercise[%
  topic=Nodes with Equal Degree,
    ]
    
    Prove that every graph $G$ with more than one node has at least two nodes with the same degree.
    
 
\subexercise[%
  topic=$3$-regular graphs,
    ]
    
    A graph is called $r$-regular if every node has degree $r$. Prove that there exists a $3$-regular graph of $n$ nodes if and only if, $n>3$ and $n$ is even.
    		
		
		\subexercise[%
  topic=Dijkstra-Algorithm,
    ]
	
	Implement the Dijkstra-Algorithm. By using it on networks of varying size $n$ verify that its time complexity is $\mathcal{O}(N^2)$ .
	

		
		%\exercise[%
  %topic=Netzwerke mit Python untersuchen
    %]

\end{document}
